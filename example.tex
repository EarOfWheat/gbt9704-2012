\documentclass{gbt9704}

\begin{document}
% --- 公文信息设置 ---
\gbDocNo{000001}			% 份号 (可选)
\gbClsLv{秘密}{65536ms}				% 密级 (可选) 
\gbUrgLv{特急}				% 紧急程度 (可选)
\gbIsuOrg{自由软件基金会,国家标准化管理委员会}			% 发文机关 (红头文字)
\gbDocNum{自软会〔2025〕10号}						% 发文字号
\gbSigner{理查德,  李华, 王小波}					% 签发人 (如果是上行文请取消注释)
\title{自由软件基金会关于做好2025年\\文化建设工作的通知} % 标题,支持换行

% 生成版头(红头、红线、标题)
\makegbstandard

% --- 主送机关 ---
% 顶格,冒号结尾
\noindent
各省、自治区、直辖市人民政府,国务院各部委、各直属机构:

% --- 正文 ---
% 自动首行缩进2字符 (模板已内置,但段落间需空行)
% \chapter{第一章总则}
GNU 是一个 自由软件 操作系统—就是说,它尊重其使用者的自由。GNU 操作系统包括 GNU 软件包(专门由 GNU 工程发布的程序)和由第三方发布的自由软件。GNU 的开发使你能够使用电脑而无需安装可能会侵害你自由的软件。

我们建议安装这些 GNU 版本(更确切地说是,GNU/Linux 发行版),它们完全是自由软件。更多关于 GNU。

\section{节标题是顿号数字}
自由软件关乎自由,而非价格。要理解这个概念,你应该考虑 “free” 是 “言论自由(free speech)”中的“自由”;而不是 “免费啤酒(free beer)”中的“免费”。

更精确地说,自由软件赋予软件使用者四项基本自由.

\subsection{标题是括号仿宋数字}
不论目的为何,有运行该软件的自由(自由之零)。

\subsubsection{次小的节是点阿拉伯数字}
有研究该软件如何工作以及按需改写该软件的自由(自由之一)。取得该软件源代码为达成此目的之前提。

\paragraph{最小的节是括号阿拉伯数字。}
有重新发布拷贝的自由,这样你可以借此来敦亲睦邻(自由之二)。

\subparagraph{段落。}
有向公众发布改进版软件的自由(自由之三),这样整个社群都可因此受惠。取得该软件源码为达成此目的之前提。


% --- 附件说明 (如有) ---
% \vspace{1em}
% \noindent 附件:1. 2025年信息化建设重点项目表

% --- 署名与日期 ---
% 右空四字 (模板命令 \gbsign{机关}{日期})
\makegbsign{自由软件基金会}{\today}

% --- 版记 ---
% \makebanji{抄送单位}{印发机关}{印发日期}
\makebanji{各人民团体}{国务院办公厅秘书局}{\zhdate{2025/12/5}}

\end{document}